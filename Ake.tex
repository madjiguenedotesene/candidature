\documentclass[a4paper,11pt, openany]{book}

% ==========================================
% 1. PACKAGES & SETUP
% ==========================================
\usepackage[utf8]{inputenc}
\usepackage[T1]{fontenc}
\usepackage[english]{babel}
\usepackage[margin=2cm]{geometry}
\usepackage{xcolor}
\usepackage{tcolorbox}
\usepackage{listings}
\usepackage{fontawesome5} % Icons
\usepackage{helvet}
\usepackage{titlesec}
\usepackage{fancyhdr}
\usepackage{graphicx}

\renewcommand{\familydefault}{\sfdefault}

% ==========================================
% 2. COLOR PALETTE (Modern Textbook)
% ==========================================
\definecolor{primary}{RGB}{0, 122, 204}    % Python Blue
\definecolor{secondary}{RGB}{255, 212, 59} % Python Yellow
\definecolor{accent}{RGB}{231, 76, 60}     % Alert Red
\definecolor{success}{RGB}{46, 204, 113}   % Green
\definecolor{dark}{RGB}{44, 62, 80}        % Dark Grey
\definecolor{codebg}{RGB}{245, 245, 245}   % Light Grey

% ==========================================
% 3. CODE STYLE
% ==========================================
\lstdefinestyle{pybook}{
    backgroundcolor=\color{codebg},
    commentstyle=\color{gray}\itshape,
    keywordstyle=\color{primary}\bfseries,
    numberstyle=\tiny\color{gray},
    stringstyle=\color{accent},
    basicstyle=\ttfamily\small\color{dark},
    breakatwhitespace=false,
    breaklines=true,
    captionpos=b,
    keepspaces=true,
    numbers=left,
    numbersep=5pt,
    showspaces=false,
    showstringspaces=false,
    showtabs=false,
    tabsize=4,
    frame=l,
    rulecolor=\color{primary},
    framesep=5pt,
    language=Python
}
\lstset{style=pybook}

% ==========================================
% 4. CUSTOM BOXES (The "Textbook" feel)
% ==========================================
\tcbuselibrary{skins,breakable}

% A. SYNTAX (Grammar Rule)
\newtcolorbox{syntaxbox}[1]{
    colback=primary!5!white,
    colframe=primary,
    title=\faCode~ \textbf{SYNTAX: #1},
    fonttitle=\bfseries,
    arc=2mm,
    boxrule=1pt,
    leftrule=5mm
}

% B. MISSION (Exercise)
\newtcolorbox{missionbox}[1]{
    colback=secondary!10!white,
    colframe=orange!80!black,
    title=\faGamepad~ \textbf{MISSION: #1},
    fonttitle=\bfseries\color{black},
    coltitle=black,
    arc=2mm,
    boxrule=1pt,
    dashed
}

% C. DEBUGGER (Solution/Tip)
\newtcolorbox{debugbox}{
    colback=dark!5!white,
    colframe=dark,
    title=\faBug~ \textbf{DEBUGGER (Correction)},
    fonttitle=\bfseries,
    arc=1mm,
    boxrule=0.5pt,
    fontupper=\small
}

% ==========================================
% 5. HEADER / FOOTER
% ==========================================
\pagestyle{fancy}
\fancyhf{}
\fancyhead[L]{\textbf{Python Level 1}}
\fancyhead[R]{Year 2025}
\fancyfoot[C]{\thepage}

% ==========================================
% 6. DOCUMENT CONTENT
% ==========================================
\begin{document}

% --- COVER PAGE ---
\begin{titlepage}
    \pagecolor{primary}
    \color{white}
    \centering
    \vspace*{4cm}
    {\Huge \faPython \par}
    \vspace{1cm}
    {\Huge \textbf{ENGLISH FOR CODERS} \par}
    \vspace{0.5cm}
    {\LARGE The Complete Python Textbook \par}
    \vspace{2cm}
    \textbf{\Large 40 Lessons $\cdot$ 8 Worlds $\cdot$ Zero to Hero}
    \vfill
    {\large "The only language you need to speak to machines."}
    \vspace{2cm}
\end{titlepage}
\pagecolor{white}

\tableofcontents
\newpage

% ==========================================
% CHAPTER 1
% ==========================================
\chapter{World 1: The Basics (Speaking)}
\textit{In this chapter, we learn how to talk to the computer.}

\section{Lesson 1-2: The Print Command}

\begin{syntaxbox}{print()}
    To display text on the screen, use \texttt{print()}.
    \textbf{Rule:} Text must be inside quotes \texttt{"..."} or \texttt{'...'}.
\end{syntaxbox}

\begin{lstlisting}[caption=Example]
print("Hello World")
print('I am a coder')
\end{lstlisting}

\begin{missionbox}{First Words}
    \textbf{Task:} Write a program that prints 3 lines:
    1. Your First Name.
    2. Your Last Name.
    3. The name of your pet (or favorite animal).
\end{missionbox}

\section{Lesson 3-5: The Calculator}

\begin{syntaxbox}{Math Operators}
    Computers are great at Math.
    \begin{itemize}
        \item \textbf{+} (Plus)
        \item \textbf{-} (Minus)
        \item \textbf{*} (Multiply)
        \item \textbf{/} (Divide)
    \end{itemize}
    \textbf{Note:} Do NOT use quotes for math!
\end{syntaxbox}

\begin{missionbox}{Space Math}
    \textbf{Task:} Calculate how many seconds are in an hour.
    \textit{Hint: 60 seconds * 60 minutes.}
\end{missionbox}

\begin{debugbox}
    \textbf{Solution:} \texttt{print(60 * 60)} \\
    \textbf{Result:} 3600 \\
    If you wrote \texttt{print("60 * 60")}, it printed the text, not the result!
\end{debugbox}

% ==========================================
% CHAPTER 2
% ==========================================
\chapter{World 2: Memory (Variables)}
\textit{How to store information in boxes.}

\section{Lesson 6-8: Creating Variables}

\begin{syntaxbox}{Variable Assignment}
    \texttt{name = value} \\
    Think of it as: "Put the \textbf{value} inside the box labelled \textbf{name}".
\end{syntaxbox}

\begin{lstlisting}
score = 100
player = "Mario"
print(player)  # Prints Mario
print(score)   # Prints 100
\end{lstlisting}

\section{Lesson 9-10: Mixing Types}
\begin{missionbox}{The Identity Card}
    \textbf{Task:}
    1. Create variable \texttt{name = "James"}.
    2. Create variable \texttt{agent\_number = 7}.
    3. Print them.
\end{missionbox}

% ==========================================
% CHAPTER 3
% ==========================================
\chapter{World 3: Interaction (Input)}
\textit{Listening to the user.}

\section{Lesson 11-13: The Input Function}

\begin{syntaxbox}{input()}
    \texttt{variable = input("Question?")} \\
    The program stops and waits for the user to type.
\end{syntaxbox}

\begin{lstlisting}
name = input("What is your name? ")
print("Hello", name)
\end{lstlisting}

\begin{missionbox}{The Parrot}
    \textbf{Task:} Create a program that asks "What is your favorite food?" and then repeats it back to you.
\end{missionbox}

\section{Lesson 14-15: Converting Numbers}
\textbf{Warning:} \texttt{input()} always gives Text. If you want a number, use \texttt{int()}.

\begin{lstlisting}
age = input("How old are you? ")
age = int(age)  # Convert to number
print("Next year you will be", age + 1)
\end{lstlisting}

% ==========================================
% CHAPTER 4
% ==========================================
\chapter{World 4: Logic (If / Else)}
\textit{Making decisions.}

\section{Lesson 16-19: If, Else}

\begin{syntaxbox}{Conditional Logic}
    \textbf{if} (condition is true): \\
    \hspace*{1cm} Do this action. \\
    \textbf{else}: \\
    \hspace*{1cm} Do the other action.
\end{syntaxbox}

\begin{missionbox}{The Bouncer}
    \textbf{Task:} Ask for age.
    If age > 18, print "Welcome to the club".
    Else, print "Go to school".
\end{missionbox}

\begin{debugbox}
\begin{lstlisting}
age = int(input("Age? "))
if age > 18:
    print("Welcome")
else:
    print("Go to school")
\end{lstlisting}
Don't forget the \textbf{colon (:)} at the end of the line!
\end{debugbox}

% ==========================================
% CHAPTER 5
% ==========================================
\chapter{World 5: Loops (Repetition)}
\textit{Don't repeat yourself. Let the machine do it.}

\section{Lesson 22-24: The While Loop}

\begin{syntaxbox}{While Loop}
    Repeats the block of code as long as the condition is True.
\end{syntaxbox}

\begin{lstlisting}
battery = 10
while battery > 0:
    print("Robot walking...")
    battery = battery - 1
print("Battery empty!")
\end{lstlisting}

\section{Lesson 25-26: The For Loop}
Best for counting things.

\begin{lstlisting}
# Count from 0 to 4
for i in range(5):
    print("Count:", i)
\end{lstlisting}

\begin{missionbox}{Punishment Generator}
    \textbf{Task:} Use a loop to print "I will not talk in class" 20 times.
\end{missionbox}

% ==========================================
% CHAPTER 6
% ==========================================
\chapter{World 6: Randomness (Modules)}
\textit{Rolling dice and luck.}

\section{Lesson 27-30: Import Random}

\begin{syntaxbox}{Modules}
    Python has tools in a toolbox. We \texttt{import} them.
\end{syntaxbox}

\begin{lstlisting}
import random
dice = random.randint(1, 6)
print("You rolled a:", dice)
\end{lstlisting}

\begin{missionbox}{Guess the Number}
    \textbf{Task (Mini-Game):}
    1. Computer picks a random number (1-10).
    2. User guesses.
    3. If correct -> Win!
\end{missionbox}

% ==========================================
% CHAPTER 7
% ==========================================
\chapter{World 7: Lists (Data)}
\textit{The Backpack inventory.}

\section{Lesson 31-35: Managing Lists}

\begin{syntaxbox}{Lists []}
    A list stores multiple items.
    Index starts at 0!
\end{syntaxbox}

\begin{lstlisting}
fruits = ["Apple", "Banana", "Cherry"]
print(fruits[0])   # Apple
fruits.append("Orange") # Add item
\end{lstlisting}

\begin{missionbox}{Shopping Cart}
    \textbf{Task:} Create a list of 3 items. Ask the user to add a 4th item. Print the full list.
\end{missionbox}

% ==========================================
% CHAPTER 8
% ==========================================
\chapter{World 8: Graphics (Turtle)}
\textit{The Final Project.}

\section{Lesson 36-40: Turtle Art}

\begin{syntaxbox}{Turtle Module}
    \texttt{import turtle} \\
    \texttt{t = turtle.Turtle()} \\
    \texttt{t.forward(100)} \\
    \texttt{t.left(90)}
\end{syntaxbox}

\begin{lstlisting}[caption=Drawing a Square]
import turtle
t = turtle.Turtle()
t.color("red")

for i in range(4):
    t.forward(100)
    t.left(90)
    
turtle.done()
\end{lstlisting}

\begin{missionbox}{FINAL PROJECT: The Star}
    \textbf{Objective:} Draw a star.
    Hint: Turn 144 degrees instead of 90.
\end{missionbox}

% ==========================================
% CONCLUSION
% ==========================================
\chapter*{Certificate}
\begin{tcolorbox}[colback=yellow!10!white, colframe=yellow!50!orange, title=CERTIFICATE OF COMPLETION, fonttitle=\bfseries\Large, center title]
    \centering
    \vspace{1cm}
    {\Large This certifies that} \\
    {\Huge \textbf{[Student Name]}} \\
    \vspace{1cm}
    {\Large Has successfully completed the} \\
    {\Large \textbf{Python Level 1 Curriculum}} \\
    \vspace{1cm}
    \faMedal \hspace{1cm} \faStar \hspace{1cm} \faMedal
\end{tcolorbox}

\end{document}